In this section various methods for maintaining a good quality of service level per tenants will be discussed.
The main challenges here are the metrics to use for measuring the current resource usage per tenant and how to seperate tenants in such a way that the behaviour of one tenant will not affect the other tenants.\cite{krebs2013metrics}

%WERKTITELS
\subsection{Why QoS Matters}
Most tenants in a multi-tenant application will have some form of Service Level Agreement (SLA) with the provider of the application.
In many cases these SLA's contain certain Quality of Service constraints the provider will have to meet, such as response time or minimum uptime. 
The tenant on the other hand will also have certain limits and/or quotas, instead of unlimited access to the application.

Based on these SLA's the available resources should be fairly distributed among tenants.
Krebs\cite{krebs2013metrics} defines a system as fair when the following criteria are met:
\begin{enumerate}
	\item Tenants who do not exceed their quota should not be affected by tenants who do.
	\item Tenants who do exceed their quota should experience performance degradation. 
		It is important to note here that even if the system is not low on resources a tenant should still get degraded performance for the system to be fair.
		This is because otherwise the tenant would be getting more than is agreed upon in their SLA, which is usually more than they are paying for and more than tenants under the same SLA who are not exceeding their quota.
	\item Tenants with a higher quote should experience better performance compared to tenants with a lower quota.
\end{enumerate}
% SLA's, eerlijk gebruik resources.

\subsection{Metrics for Tenant Load}
%ZIE: QoS 5.: Krebs et.al.  QoS 2.: Kwok et.al.

\subsection{Tenant Seperation Techniques}
%TODO: Let op: in hoeverre verschillende architectuurkeuzes al behandeld in het algemene deel.
% 1. VMs (tenants toekennen aan machines)
% 2. Tenants toekennen aan specifieke resources (thread pools e.d.) 
% 3. Alle tenants delen dezelfde resources (QoS via: request throttling / queueing.)

\subsection{Current Challenges in QoS Research}
