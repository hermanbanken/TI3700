%INTRO: (section title is in main file.)
%\subsection{scalability.tex}
The scalability of a system, in general, describes how well a system can handle increasing workload, stored data and utilisation. 
A system is considered scalable when it can be easily extended to support an increasing load. 
For an unscalable system supporting increased load might be technically infeasable or prohibitively expensive.\cite{bondi2000scalability}

In the scalability section we will focus specifically on how a multi-tenant application can deal with an increasing number of tenants, increasing amount of data and an increasing amount of data variability. 
Seperation of tenants and maintaining performance during peak usage will be discussed in the QoS section.

%WERKTITELS!
\subsubsection{Where and how Scalability Matters}
As one of the key advantages of multi-tenant applications is a higher utilisation of resources and the associated cost reduction per tenant.\cite{bezemer2010multi} 
It is required that a multi-tenant application is scalable to maintain this cost reduction and to accomodate for the, hopefully, ever increasing number of tenants.
This also means that it is both undesirable to underprovision resources and to overprovision resources. The former causing performance degradation, the latter causing increased and unneeded costs.

%TODO: 	Uitleg research richtingen, plaatsing van tenants, hoe/WANNEER moet ik opschalen als er meer tenants bij komen
%	  	Uitleg DB onderzoeken. 

\subsubsection{Scalability in the data layer}
A database for a multi-tenant application should be extendable bij the tenants.
This allows for tenant specific modifications to the schema.
%TODO: Kijken in hoeverre dit al onder het algemene deel behandeld is.

Thus, scalability in the data layer is important in two different ways.
First, as the number of tenants increases so will the amount of data and load on the database.
Second, as the number of tenants increases so will the amount of customisations to the schema. 
The database schema should be managing this as effici\"ently as possible.

In the sections after this we will look at various schema mapping techniques and their pro's \ cons.
After we will discuss two prototype tenant aware DBMSes.

\subsubsection{Schema mapping techniques}
\begin{description}
	\item[Universal table(s)]
	\item[Private tables]
	\item[Sparse columns]
	\item[Extension field]
	\item[Extension tables]
	\item[Chunk folding] %Includes chunk tables! 
	\item[Pivot table]
\end{description}

\subsubsection{Tenant aware DMBSes}
%FLEXSCHEME and the tenant aware RDBMS.

\subsubsection{Resource allocation for new tenants}
%Allocation of resources for new tenants, predicting & preventing conflicts / bottlenecks.

\subsubsection{Discussion of challenges \& future work}
%Spreekt voor zich.
