%INTRO: (section title is in main file.)
%\subsection{scalability.tex}
The scalability of a system, in general, describes how well a system can handle increasing workload, stored data and utilisation.
A system is considered scalable when it can be extended to support an increasing load with a wel manageable cost increase.
On the contrary, for an unscalable system supporting increased load might be technically infeasable or prohibitively expensive.\cite{bondi2000scalability}

As one of the key advantages of multi-tenant applications is a higher utilisation of resources and the associated cost reduction per tenant.\cite{bezemer2010multi} 
It is required that a multi-tenant application is scalable to maintain this cost reduction and to accomodate for the, hopefully, ever increasing number of tenants.

In the scalability section we will focus specifically on how the infrastructure, or specific parts thereof, of a multi-tenant application can deal with an increasing number of tenants.

%WERKTITELS!
\subsubsection{Where and how Scalability Matters}

%TODO: 	Uitleg research richtingen, plaatsing van tenants, hoe/WANNEER moet ik opschalen als er meer tenants bij komen
%	  	Uitleg DB onderzoeken. 

\subsubsection{Scalability in the data layer}
The schema of the datamodel presented in a multi-tenant application should be extensible by tenants.
Extensibility is required because this allows tenants to modify their view of the application in such a way that a better fits their needs.
%TODO: Kijken in hoeverre dit al onder het algemene deel behandeld is.

Thus, scalability in the data layer is important in two different ways.
First, as the number of tenants increases so will database load.
Second, as the number of tenants increases so will the amount of extensions on the the schema. 
The database should manage both the increasing load and the increasing complexity of the data in a scalable way.

Scaling a database under increasing load is not a multi-tenant specific problem and for most DBMSes solutions exist for this. 
For instance, Postgresql has several available solutions for scaling and clustering\footnote{http://wiki.postgresql.org/wiki/Replication,\_Clustering,\_and\_Connection\_Pooling (Checked march 2014)}, and so does MySQL.\footnote{https://www.mysql.com/products/cluster/scalability.html (Checked march 2014)}

This is also reflected in the fact that most research on performance and scalability in the data layer for multi-tenant applications either focuses on the following two subjects.
How to efficiently map the extensible schema onto existing RDBMSes.\cite{aulbach2008multi, aulbach2009comparison} 
Or, more recently, on creating multi-tenant aware RDBMSes.\cite{schiller2011native, aulbach2011extensibility} 

\subsubsection{Schema mapping techniques}
Schema mapping techniques for multi-tenant applications van be seperated seperated into two groups. 
One group in wich the database 'owns' the schema  and the other group where the application 'owns' the schema and maps this into generic database structures.\cite{aulbach2009comparison}

The first group lies closest to traditional usage of an RDBMS. 
The various entities in the schema generally have their own tables in the database.
Below several schema layouts in this group will be discussed.
%TODO: Performance characteristics per schema.
\begin{description}
	\item[Basic layout] In the basic layout the database fully owns the schema and it cannot be extended. 
		In this schema the tables are shared among tenants and the tables contain a 'tenantid' column to seperate the date of the various tenants.
		In this case there is no room for extensions to the schema. 
		This schema type is common in less complex applications where extensibility is less needed. \cite{aulbach2008multi}
	\item[Private tables] Using a private tables layout every tenant has his own set of tables. 
		Naturally this allows for tenants to extend the schema. 
		One major drawback is that this layout will only scale up to a certain point. 
		A large amount of tables will lead to performance degradation.\cite{aulbach2008multi}
	\item[Extension tables]
		This layout combines the the basic and the private tables layout.
		In this schema there is a base table for every entity shared across tenants with a rowid and tenantid column.
		When the schema is extended an extension table with the additional columns is created.
		This table will also contain a rowid and tenantid column wich are used to reconstruct the data into its extended form.
		In the optimal case existing extensions can be used by other tenants as well, leading to a reduced number of tables per tenant compared to the previous private layout.\cite{aulbach2008multi}
\end{description}

In the mappings discussed below the database does not own the schema. 
For most layouts below this means that knowledge of column names and types are kept in the application or middleware that is using the database for storage. 
\begin{description}
	\item[Universal table(s)]
		The universal table is a schema layout where there exists one generic table with a tenant column, table column and a number of generic data column.
		Onto this table several logical entities can be mapped.
		In this mapping the n'th column of the logical entity will map to the n'th generic field of its row in the universal table.
		The application or middleware will have to keep track of the mappings for the entities used in the application.\cite{aulbach2008multi}

		%TODO: Pivot tables
		%Een tabel voor alles, met vast aantal custom fields (varchar oid) eraan
	\item[Extension field]
		In the extension field layout for every entity a table.
		All tables of extensible entities contain an extra field that stores extensions in a structured format like XML or JSON.
		Certain RDBMSes support accessing data within such a field. 
		DB2 does this for XML trough PureXML\footnote{http://www-01.ibm.com/software/data/db2/linux-unix-windows/xml/index.html (march 2014)} and Postgres allows for reading values from json fields\footnote{http://www.postgresql.org/docs/9.3/static/functions-json.html (march 2014)}
	
	\item[Chunk tables] 
		%TODO: overwegen deze echt op te nemen of te noemen dat er hybrides bestaan.
		%Includes chunk tables! 
		%Like extension tables however customisations are stored in seperate chunk tables, wich can contain customisations for multiple tenants and even multiple tables.
\end{description}

\subsubsection{Tenant aware DMBSes}
%FLEXSCHEME and the tenant aware RDBMS.

\subsubsection{Resource allocation for new tenants}
%Allocation of resources for new tenants, predicting & preventing conflicts / bottlenecks.
% Espadas et.al en Thomas Kwok et.al.

\subsubsection{Discussion of challenges \& future work}
%Spreekt voor zich.
