Similair to the concept of multi-tenancy itself, the security aspect of multi-tenancy is relatively new and open to more research. 
As discussed in other sections, the model offers several significant advantages in comparison to other software models. 
While the cloud offers advantages, until some of the risks are better understood, many of the major players will be tempted to hold back. 
To advance the development and deployment of cloud computing, security issues and approaches for the techniques range and cross various technical domains including cryptography, virtualization and programming~\cite{Takahashi2012Security}. 

The importance of security is particularly apparent in the adoption process of the multi-tenant model. 
Many potential businesses which would be interested in cloud computing are still reluctant to adopt it due to security to security and privacy concerns, according to Bernabe et al~\cite{Bernabe2012Auth}. 
The concept of multi-tenancy consists of having tenants share IT infrastructure and the sharing of resources. 
Though cloud computing is targeted to provide better utilization of resources using virtualization techniques and to take up much of the work load from the client, it is fraught with security risks, following Seccombe et al~\cite{Seccombe2009Security}. 
According to Takahashi~\cite{Takahashi2012Security} this implies the design of strong security boundaries in order to isolate tenants when using these shared resources. 
Thus, the system to control the access to the available resources becomes a critical aspect in order to provide an efficient control over the usage of the cloud architecture.

Next to technical security issues, privacy questions also need to be addressed to realize the potential of this new model. 
Cloud computing moves the application software and databases to the large data centers, where management of the data and services are not trustworthy. 
This unique attribute poses many new security challenges, according to Cong Wang et al~\cite{Wang2009Security}. 

\subsection{Security Issues with Multi-Tenancy}
In this section we will elaborate on the security aspect of the multi-tenancy model. 
First, we will discuss security issues in the multi-tenancy model.

Secondly, in the following sections some of the security issues closely linked to multi-tenancy, such as on the aspect of virtualization, will be explained. 
Finally we list the remaining challenges and future work in the final section.
% Needs revision

\subsubsection{Localization}
One of the data confidentiality issues with multi-tenancy systems is the issue regarding the physical location of the data. 
Although it is easy to forget, all the data stored by the clients in a multi-tenant system still has to be stored somewhere on a physical location. 
The choice of the physical company can cause a lot of privacy and legal issues. 
According to Softlayer~\cite{Softlayer2009Security}, compliance and data privacy laws in various countries locality of data are of utmost importance in many enterprise architectures. 
For example, in many European and South American countries, certain types of data cannot leave the country because of potentially sensitive information. 
Besides these restrictions, there is also the issue of jurisdiction.
The often difficult question is raised which locality has the jurisdiction over the data, when an investigation would occur.
As stated in~\cite{Subashini2011Security}, a secure model must be capable of providing reliability to the customer regarding on the location of the data of the customer.
% Meer over onderzoek

\subsubsection{Secure Data Storage}
One of the most important characteristics of a good multi-tenant system is that tenants are only able to view and modify their own data. This is a relatively difficult security issue for these systems, due to the fact that all tenants share the same application functionality and databases. Malicious tenants could try to use loop holes to hack their way to access of the data of other tenants. Tenants are often allowed to add custom code to these services, which makes the risk of data intrusion even bigger when precautions aren’t properly taken. A multi-tenant model should therefore ensure a clear ‘firewall’ for each tenant’s data. The boundary must be ensured not only at physical level but also at the application level, as stated in the paper by Subashini~\cite{Subashini2011Security}.
%[Kan nog wat over data redundancy cancellation and backup)

To ensure the secure data storing, the paper of Takahashi et al~\cite{Takahashi2012Security} suggests the use of encrypted data manipulation using crypto-graphical techniques. 
One such technique is called \acf{PECE}, which allows a user to encrypt a file with multiple keys, while decrypting it with a single key. 
Next to that, they also propose Homomorphic encryption. 
This crypto-graphical technique enables the users to perform operations on encrypted files without decrypting the value.
% Refactoren naar simpelere uitleg

\subsubsection{Authentication and Authorization}
The matter of authentication and authorization, often referred to as data access control, in multi-tenancy has been discussed extensively over the last years. 
Simple authentication schemes, where a user either has or has no access to all content, are widely available. According to Bernabe et al~\cite{Bernabe2012Auth} current cloud providers such as Rackspace\footnote{http://www.rackspace.com/} or Amazon EC2\footnote{http://aws.amazon.com/ec2/} only rely on simple authentication schemes which do not provide enhanced access control capabilities. 
These systems often lack the functionality and complexity to express more advanced forms of authorization. 
Thus, the problem lies with designing more advanced schemes of authentication, where users can be granted custom privileges, besides having full administrator access.

Noteworthy progress can be noted in Bernabe~\cite{Bernabe2012Auth}, which proposes an access control model system suitable for multi-tenancy and grants high expressiveness in terms of permissions. 
This expressiveness is also supported by the integration of semantic web technologies into the authorization model. 
The system allows a fine-grained definition of what resources should available for each particular tenant. 
Another influence in this area is the system proposed in Calero~\cite{Calero2010Auth}. 
This authorization system is able to support collaboration agreements, often referred to as federations, between tenants or businesses.

\subsubsection{Web Application Security}
The web is an inevitable aspect of the multi-tenancy model. 
It ensures the accessibility of the data. 
The importance of good security practices in the application-layer is be illustrated by \highlight{(Wade et al, 2008)}. 
The report about data breaches on the Verizon Business platform, reports the following distribution of breaches:

\begin{enumerate}
    \item Application/service layer – 39\%
    \item OS/platform layer – 23\%
    \item Exploit known vulnerability – 18\%
    \item Exploit unknown vulnerability – 5\%
    \item Use of back door – 5\%
\end{enumerate}

The statistics are quite clear about it; nowadays hackers and other malicious users tend to attack the higher layers, such as the application-layer. 
The \ac{OWASP}\footnote{http://owasptop10.googlecode.com/files/OWASP\%20Top\%2010\%20-\%202013.pdf} has identified the 10 greatest security risks faced by web applications, including SQL injections and cross-server scripting.

In Takahashi et al~\cite{Takahashi2012Security} the authors describe a couple of ways to detect vulnerabilities in the server-side and client-side of the web application. 
Endpoint risk detection techniques detect client-side vulnerabilities at the endpoint (the user). 
There are a good number of implementations, such as FLAX~\cite{saxena10kudzu} and Zozzle~\cite{curtsinger2011zozzle} that target JavaScript issues. 
Another form of detection is called the middle-box risk detection. 
This kind of detection requires no adjustments of the code, as the detection is performed in between the server and client-side, using custom HTTP requests. 
Projects implementing this kind detection, such as SpyProxy, BrowserShield and WebShield, are still in early development stages, but they already look very promising.

\subsubsection{Data Integrity and Network Security}
With the dependence on extensive usage of networks, the multi-tenancy model is also highly dependent on good network security.  
The data transmissions over the network needs to be secured to prevent loss or theft of classified information. 
According to Subashini~\cite{Subashini2011Security}, one of the biggest challenges with multi-tenant services is transaction management. 
At the protocol level, HTTP does not offer any support for transaction or guaranteed delivery of packets. 
Thus, to ensure transactions are indeed delivered one needs to implement this functionality into the multi-tenancy system.

Currently there are some standards available trying to fix this security issue, namely WS-Transaction\footnote{http://msdn.microsoft.com/en-us/library/ms951262.aspx} and WS-Reliability.
However, as noted by Subashini et al.~\cite{Subashini2011Security}, these standards haven’t reached technical maturity yet and have therefore yet to experience full adoption by the majority of the multi-tenancy providers.
Since the publishing of the paper by Subashini.~\cite{Subashini2011Security}, WS-reliability has since been superseded by ReliableMessaging\footnote{http://docs.oasis-open.org/ws-rx/wsrm/200702/wsrm-1.1-spec-os-01.pdf}.
% Bewijs minimale adoptie

\subsection{Related Security Issues}
Taking a wider scope on the subject, papers indicate a lot of security issues closely linked to multi-tenancy. 
These security issues should be taken into account due to the following two reasons.

First off, as mentioned earlier, the definition of multi-tenancy is still quite ambiguous. 
It is often used to indicate all sorts of cloud services, including IaaS and PaaS models, as seen by Jasti et al~\cite{Jasti2010Security}. 
The scope of multi-tenancy is often larger than our definition of multi-tenancy. Due to the increased scope, more security issues can be considered to belong to multi-tenancy.

Another reason for taking into account related security issues, is the fact that multi-tenancy is a high-level model. 
The model depends on a bundle of underlying technologies, such as the hardware infrastructure, operating systems and server software. 
Each of these technologies has its own share of security issues, impacting the level of security of the multi-tenancy system on top. 
% Discuss related issues (which will not be treated

\subsubsection{\acf{VM} Security}
When taking virtual machines into account in multi-tenancy. 
Introspection of virtual machines pose major challenges to the cloud provider. 
A tenant has the ability to migrate his custom \acp{VM} to other \acp{VMM}. 
However, when a \ac{VM} is placed in a server with an untrusted \ac{VMM} it would allow \ac{VMM} to track to the data flows inside the guest VM~\cite{Takahashi2012Security}.

Another current issue is the adaptation of VM-based Root-kits. 
These root-kits, unlike traditional root-kits, do not stop at OS level, but continue to attempt to infect the supervising \ac{VMM}. 
According to Takahashi~\cite{Takahashi2012Security}, several proof-of-concept VM-based root-kits, such as Blue Pill, were able to successfully identify and infect the VM, followed by the \ac{VMM}.

\subsubsection{Virtual Machine Monitor Security}
The \acl{VMM}, often referred to as the hypervisor, has the essential role of isolating and controlling the virtual machines. 
However, an investigation was conducted by Ormandy et al~\cite{Ormandy2007Security}of six major \acp{VMM} and emulators, using source code auditing techniques. All six systems had major flaws, leading to unexpected aborts and possible exploits.

Additionally there is the problem of the detectability of the \ac{VMM}’s. Ideally, the \ac{VMM} is completely transparent; the tenant has no notion what kind of \ac{VMM} is running the virtual machines. 
However, as argued in the paper by Takahashi~\cite{Takahashi2012Security}, that the idea of \ac{VMM} transparency is unrealistic, due to a significant number of implicit clues, such as time sources and overhead, given by the system. 
The detectability of the \ac{VMM} creates the opportunity for malicious users to target specific \ac{VMM} systems and versions.

\subsection{Research Agenda for Security}
As mentioned in previous sections the security aspect of multi-tenancy is relatively new.
The survey of the literature regarding security, revealed the following recommendations for prospective researchers to look in to. 
\begin{itemize}
    \item Merino~\cite{Merino2011Security} suggests research to be done on implementation specific security risks. In particular they suggest more research on methods on how to stop non-trusted threads without affecting the platform safely. Furthermore the paper proposes more research on mechanisms that allow resource sharing policies to be properly implemented.
    \item According to Takahashi~\cite{Takahashi2012Security} more research is needed in general on some of the technically immature areas, such as the security of the web-layer and hypervisor-layer.
Those areas of security are relatively new and, although they have some challenging issues, haven't got any good, realistic solutions yet.
    \item Bernabe~\cite{Bernabe2012Auth} proposes an intensive analysis of the proposed authentication model.
Furthermore, it looks for an analysis and comparison of other trust models and their suitability for cloud computing.
    \item According to Calero~\cite{Calero2010Auth} we need to have more research on more advanced authorization models, next to having more experimentation with different database-systems for the proposed authorization system.
    \item Hashizume et al~\cite{Hashizume2013Security} suggests more investigative research on some of the incomplete issues discussed in the paper.
\end{itemize}
To summarize, there is a lot research to be done on the security issues of multi-tenancy systems.
Particularly, the papers in the survey aim for more research on practical implementations of the mostly theoretical solutions proposed.
% Restructuring of research agenda
