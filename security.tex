\subsection{Security in Multi-Tenancy}
Like the concept of multi-tenancy itself, the security aspect of multi-tenancy is relatively new and open to more research. As discussed in other sections, the model offers several significant advantages in comparison to other software models. While the cloud offers advantages, until some of the risks are better understood, many of the major players will be tempted to hold back. To advance the development and deployment of cloud computing, security issues and approaches for the techniques range and cross various technical domains including cryptography, virtualization and programming \cite{Takahashi2012Security}.\\
% Could be left out... ^
% These two sections could be merged..

\subsection{Importance of Security}
The importance of decent security is particularly apparent in the adoption process of the multi-tenant model. Many potential businesses which would be interested in cloud computing are still a bit reluctant to adopt it due to security to security and privacy concerns, according to Bernabe et all\cite{Bernabe2012Auth}. The concept of multi-tenancy consists of having tenants share IT infrastructure and the sharing of resources. Though cloud computing is targeted to provide better utilization of resources using virtualization techniques and to take up much of the work load from the client, it is fraught with security risks (Seccombe et al, 2009). According to Takahashi\cite{Takahashi2012Security} this implies the design of strong security boundaries in order to isolate tenants when using these shared resources. Thus, the system to control the access to the available resources becomes a critical aspect in order to provide an efficient control over the usage of the cloud architecture. \\
Next to technical security issues, privacy questions also need to be addressed to realize the potential of this new model. Cloud computing moves the application software and databases to the large data centers, where management of the data and services are not trustworthy. This unique attribute poses many new security challenges (Cong Wang et al, 2009). \\

\subsection{Security Issues with Multi-Tenancy}
In the following sections we will expand upon the security aspect in the multi-tenancy model. First of all, in first sections we will discuss current security issues in the multi-tenancy model. \\
Secondly, in the following sections some of the security issues closely linked to multi-tenancy, such as on the aspect of virtualization, will be explained. Finally we list the remaining challenges and future work in the final section\\

\subsubsection{Localization}
One of the data confidentiality issues with multi-tenancy systems is the issue regarding the physical location of the data. Although it is easy to forget, all the data stored by the clients in the cloud still has to be stored somewhere on a physical location. The choice of the physical company can cause a lot of privacy and legal issues. According to (Softlayer, 2009), compliance and data privacy laws in various countries locality of data are of utmost importance in many enterprise architectures. For example, in many European and South American countries, certain types of data cannot leave the country because of potentially sensitive information. Besides these restrictions, there is also the issue of jurisdiction. The often difficult question is raised which locality has the jurisdiction over the data, when an investigation would occur. As stated in \cite{Subashini2010Security}, a secure model must be capable of providing reliability to the customer regarding on the location of the data of the customer.\\

\subsubsection{Secure Data Storage}
One of the most important characteristics of a good multi-tenant system is that tenants are only able to view and modify their own data. This is a relatively difficult security issue for these systems, due to the fact that all tenants share the same application functionality and databases. Malicious tenants could try to use loop holes to hack their way to access of the data of other tenants. Tenants are often allowed to add custom code to these services, which makes the risk of data intrusion even bigger when precautions aren’t properly taken. A multi-tenant model should therefore ensure a clear ‘firewall’ for each tenant’s data. The boundary must be ensured not only at physical level but also at the application level, as stated in the paper by Subashini \cite{Subashini2010Security}. \\
%[Kan nog wat over data redundancy cancellation and backup)
To ensure the secure data storing, the paper \cite{Takahashi2012Security} suggests the use of encrypted data manipulation using cryptographical techniques. One such technique is called progressive elliptic curve encryption (PECE), which allows a user to encrypt a file with multiple keys, while decrypting it with a single key. Next to that, they also propose Homomorphic encryption. This cryptographical technique enables the users to perform operations on encrypted files without decrypting the value.\\

\subsubsection{Authentication and Authorization}
The matter of authentication and authorization, also often referred to as data access control, in multi-tenancy has been discussed extensively over the last years. Simple authentication schemes, where a user either has or has no access to all content, are widely available. According to Bernabe et all \cite{Bernabe2012Auth} current cloud providers such as Rackspace \footnote{http://www.rackspace.com/} or Amazon EC2 \footnote{http://aws.amazon.com/ec2/} only rely on simple authentication schemes which do not provide enhanced access control capabilities. These systems often lack the functionality and complexity to express more advanced forms of authorization. Thus, the problem lies with designing more advanced schemes of authentication, where users can be granted custom privileges, besides having full administrator access.\\
Noteworthy progress can be noted in bernabe\cite{Bernabe2012Auth}, which proposes an access control model system suitable for multi-tenancy and grants high expressiveness in terms of permissions. This expressiveness is also supported by the integration of semantic web technologies into the authorization model. The system allows a fine-grained definition of what resources should available for each particular tenant. Another influence in this area is the system proposed in Calero\cite{Calero2010Auth}. This authorization system is able to support collaboration agreements, often referred to as federations, between tenants or businesses.\\

\subsubsection{Web Application Security}
The web is an inevitable aspect of the multi-tenancy model. It ensures the accessibility of the data. The importance of good security practices in the application-layer can be illustrated by (Wade et al, 2008). The report about data breaches on the Verizon Business platform, where the following:\\
\begin{enumerate}
    \item Application/service layer – 39\%
    \item OS/platform layer – 23\%
    \item Exploit known vulnerability – 18\%
    \item Exploit unknown vulnerability – 5\%
    \item Use of back door – 5\%
\end{enumerate}
The statistics are quite clear about it; nowadays hackers and other malicious users tend to attack the higher layers, like the application-layer. The Open Web Application Security Project (OWASP) \footnote{http://owasptop10.googlecode.com/files/OWASP\%20Top\%2010\%20-\%202013.pdf} has identified the 10 greatest security risks faced by web applications, including SQL injections and cross-server scripting. \\
In Takahashi et all\cite{Takahashi2012Security} the authors describe a couple of ways to detect vulnerabilities in the server-side and client-side of the web application. Endpoint risk detection techniques detect client-side vulnerabilities at the endpoint (the user). There are a good number of implementations, such as FLAX and Zozzle that target JavaScript issues. Another form of detection is called the middlebox risk detection. This kind of detection requires no adjustments of the code, as the detection is performed in between the server and client-side, using custom HTTP requests. Projects implementing this kind detection, such as SpyProxy, BrowserShield and WebShield, are still in early development stages, but they already look very promising.\\

\subsubsection{Data Integrity and Network Security}
With the dependence on extensive usage of networks, the multi-tenancy model is also highly dependent on good network security.  The data transmissions over the network needs to be secured to prevent loss or theft of classified information. According to Subashini\cite{Subashini2011Security}, one of the biggest challenges with multi-tenant services is transaction management. At the protocol level, HTTP does not offer any support for transaction or guaranteed delivery of packets. Thus, to ensure transactions are indeed delivered one needs to implement this functionality into the multi-tenancy system.\\
	Currently there are some standards available trying to fix this security issue, namely WS-Transaction \footnote{http://msdn.microsoft.com/en-us/library/ms951262.aspx} and WS-Reliability, which has been superseded by ReliableMessaging \footnote{http://docs.oasis-open.org/ws-rx/wsrm/200702/wsrm-1.1-spec-os-01.pdf}. However, noted by (subashini, 2011), these standards haven’t reached technical maturity yet and have therefore yet to experience full adoption by the majority of the multi-tentancy providers.\\

\subsection{Related Security Issues}
Taking a wider scope on the subject, papers indicate a lot of security issues closely linked to multi-tenancy. These security issues should be taken into account due to the following two reasons.\\
First off, as mentioned earlier, the definition of multi-tenancy is still quite ambiguous. It is often used to indicate all sorts of cloud services, including IaaS and PaaS models, as seen by Jasti et all\cite{Jasti2010Security}. The scope of multi-tenancy is often larger than our definition of multi-tenancy. Due to the increased scope, more security issues can be considered to belong to multi-tenancy. \\
Another reason for taking into account related security issues, is the fact that multi-tenancy is a high-level model. The model depends on a bundle of underlying technologies, such as the hardware infrastructure, operating systems and server software. Each of these technologies has its own share of security issues, impacting the level of security of the multi-tenancy system on top. \\

\subsubsection{Virtual Machine Security}
When taking virtual machines into account in multitenancy. Introspection of virtual machines pose major challenges to the cloud provider. A tenant has the ability to migrate his custom VMs to other VMMs. However, when a VM is placed in a server with an untrusted VMM it would allow VMM to track to the data flows inside the guest VM \cite{Takahashi2012Security}. \\
Another current issue is the adaptation of VM-based Rootkits. These rootkits, unlike traditional rootkits, do not stop at OS level, but continue to attempt to infect the supervising VMM. According to Takahashi\cite{Takahashi2012Security}, several proof-of-concept VM-based rootkits, such as Blue Pill, were able to successfully identify and infect the VM, followed by the VMM.\\

\subsubsection{Virtual Machine Monitor Security}
The Virtual Machine Monitor (VMM), often referred to as the hypervisor, has the essential role of isolating and controlling the virtual machines. However, an investigation was conducted by Ormandy et all\cite{Ormandy2007}of six major VMMs and emulators, using source code auditing and fuzzing techniques. All six systems had major flaws, leading to unexpected aborts and possible exploits.\\
Additionally there is the problem of the detectability of the VMM’s. Ideally, the VMM is completely transparent; the tenant has no notion what kind of VMM is running the virtual machines. However, as argued in the paper by Takahashi\cite{Takahashi2012Security}, that the idea of VMM transparency is unrealistic, due to a significant number of implicit clues, such as time sources and overhead, given by the system. The detectability of the VMM creates the opportunity for malicious users to target specific VMM systems and versions.\\

\subsubsection{Platform Security}
Major platforms, such as the Java platform, also contain certain limitations that hinder the construction of secure multi-tenant systems. Resource accounting is one of those relevant issue on platforms. The security around the usage of resources is, as explained in other issues, rather simplified. Once access is granted to some code, that code can use the resource without limitations, as argued by Merino\cite{Merino2011Security}. There is no accounting of resource usage by threads. A hostile user could try to create memory-overflows by creating many instances of objects.
An issue present in the major platforms is safe thread termination. This problem is due to the lack of a safe way to enforce thread termination and is present in Java and .NET platforms \cite{Merino2011Security}. Aborting a thread in Java and .NET languages can be thought of as a strongly worded suggestion to the thread that it should stop functioning; a thread running untrusted code can just ignore it and continue executing possibly hostile code.\\

\subsubsection{OS Layer Security}
Like all the stacked layers of software, integrity of the OS is a major concern. As stated in the paper by Takahashi\cite{Takahashi2012Security} the chain of trust should be assured at every layer of the software stack. With the execution of possible untrusted software on top of the OS, creates the possibility for hackers to exploit known vulnerabilities of the operating system to gain access to the complete system, when security preventing these practices is lacking.\\
% Could be left out; not very interesting...


\subsection{Future Work on Security}
As mentioned in previous sections the security aspect of Multi-Tenancy is relatively new. The survey of the literature regarding security, revealed the following recommendations for prospective researchers to look in to. \cite{Moreno2011Security} suggests research to be done on implementation specific security risks. In particular they suggest more research on methods on how to stop non-trusted threads without affecting the platform safely. Furthermore the paper proposes more research on mechanisms that allow resource sharing policies to be properly implemented. More research by Takahashi\cite{Takahashi2012Security} is needed in general on some of the technically immature areas, such as the security of the web-layer and hypervisor-layer. Those areas of security are relatively new and, although they have some challenging issues, haven't got any good, realistic solutions yet. Bernabe\cite{Bernabe2012Auth} proposes an intensive analysis of the proposed authentication model. Furthermore, it looks for an analysis and comparison of other trust models and their suitability for cloud computing. According to Calero \cite{Calero2010Auth} we need to have more research on more advanced authorization models, next to having more experimentation with different database-systems for the proposed authorization system. Hashizume et all\cite{Hashizume2013Security} suggests more investigative research on some of the incomplete issues discussed in the paper. In [6] it is announced that the authors are looking into further improvement in areas such as authentication and the protection against VM attacks. To summarize, there is a lot research to be done on the security issues of multi-tenancy systems. Particularly, the papers in the survey aim for more research on practical implementations of the mostly theoretical solutions proposed.\\
% future work of 3 papers missing.
