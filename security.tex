- Security in Multi-Tenancy.
Like the concept of multi-tenancy itself, the security aspect of multi-tenancy is relatively new and open to more research. As discussed in other sections, the model offers several significant advantages in comparison to other software models. While the cloud offers advantages, until some of the risks are better understood, many of the major players will be tempted to hold back (viega, 2009). To advance the development and deployment of cloud computing, security issues and approaches for the techniques range and cross various technical domains including cryptography, virtualization and programming [enabling secure mt..].

- The importance of security
The importance of decent security is particularly apparent in the adoption process of the multi-tenant model. Many potential businesses which would be interested in cloud computing are still a bit reluctant to adopt it due to security to security and privacy concerns, according to [future generation..]. The concept of multi-tenancy consists of having tenants share IT infrastructure and the sharing of resources. Though cloud computing is targeted to provide better utilization of resources using virtualization techniques and to take up much of the work load from the client, it is fraught with security risks (Seccombe et al, 2009). According to [future generation…], this implies the design of strong security boundaries in order to isolate tenants when using these shared resources. Thus, the system to control the access to the available resources becomes a critical aspect in order to provide an efficient control over the usage of the cloud architecture. 
Next to technical security issues, privacy questions also need to be addressed to realize the potential of this new model. Cloud computing moves the application software and databases to the large data centers, where management of the data and services are not trustworthy. This unique attribute poses many new security challenges (Cong Wang et al, 2009). Issues, such as data protection, resource isolation and communication security, need to be addressed to convince the majority of potential users of converting to this software model.

- Security issues with multi-tenancy
In the following sections we will expand upon the security aspect in the multi-tenancy model. First of all, in sections [SECTIONS] we will discuss current security issues in the multi-tenancy model. 
Secondly, in the sections [SECTIONS] some of the security issues closely linked to multi-tenancy, such as on the aspect of virtualization, will be explained. To continue, in section XXX.2 we summarize some of the solutions to the mentioned issues. Finally we list the remaining challenges and future work in section XXX.3.

+ Localization
One of the least technical issues with multi-tenancy systems is the issue regarding the physical location of the data. Although it is easy to forget, all the data stored by the clients in the cloud still has to be stored somewhere on a physical location. The choice of the physical company can cause a lot of privacy and legal issues. According to (Softlayer, 2009), compliance and data privacy laws in various countries locality of data are of utmost importance in many enterprise architectures. For example, in many European and South American countries, certain types of data cannot leave the country because of potentially sensitive information. Besides these restrictions, there is also the issue of jurisdiction. The often difficult question is raised which locality has the jurisdiction over the data, when an investigation would occur. As stated in (Subashini, 2010), a secure model must be capable of providing reliability to the customer regarding on the location of the data of the customer.
 
+ Secure data storing
One of the most important characteristics of a good multi-tenant system is that tenants are only able to view and modify their own data. This is a relatively difficult security issue for these systems, due to the fact that all tenants share the same application functionality and databases. Malicious tenants could try to use loop holes to hack their way to access of the data of other tenants. Tenants are often allowed to add custom code to these services, which makes the risk of data intrusion even bigger when precautions aren’t properly taken. A multi-tenant model should therefore ensure a clear ‘firewall’ for each tenant’s data. The boundary must be ensured not only at physical level but also at the application level, as stated in (Subashini, 2012). [Kan nog wat over data redundancy cancellation)
To ensure the secure data storing, the paper (takashi, 2012) suggests the use of encrypted data manipulation using cryptographical techniques. One such technique is called progressive elliptic curve encryption (PECE), which allows a user to encrypt a file with multiple keys, while decrypting it with a single key. Next to that, they also propose Homomorphic encryption. This cryptographical technique enables the users to perform operations on encrypted files without decrypting them.

+ Authentication and authorization
The matter of authentication and authorization in multi-tenancy has been discussed extensively over the last years. Simple authentication schemes, where a user either has or hasn’t access to all content, are widely available. According to (Bernabe, 2012) current cloud providers such as Rackspace or Amazon EC2 only rely on simple authentication schemes which do not provide enhanced access control capabilities. These systems often lack the functionality and complexity to express more advanced forms of authorization. Thus, the problem lies with designing more advanced schemes of authentication, where users can be granted custom privileges, besides having full administrator access.
Noteworthy progress can be noted in (Bernabe, 2012), which proposes an access control model system suitable for multi-tenancy and grants high expressiveness in terms of permissions. This expressiveness is also supported by the integration of semantic web technologies into the authorization model. The system allows a fine-grained definition of what resources should available for each particular tenant. Another influence in this area is the system proposed in (Calero, 2010). This authorization system is able to support collaboration agreements, often referred to as federations, between tenants or businesses.

+ Web application security
The web is a truly inevitable aspect of the multi-tenancy model. The importance of good security practices in the application-layer can be illustrated by (Wade et al, 2008). The report about data breaches on the Verizon Business platform, where the following:
* Application/service layer – 39%
* OS/platform layer – 23%
* Exploit known vulnerability – 18%
* Exploit unknown vulnerability – 5%
* Use of back door – 5%
The statistics are quite clear about it; nowadays hackers and other malicious users tend to attack the higher layers, like the application-layer. The Open Web Application Security Project (OWASP) [footnote!] has identified the 10 greatest security risks faced by web applications, including SQL injections and cross-server scripting. 
In (takashi, 2012) the authors describe a couple of ways to detect vulnerabilities in the server-side and client-side of the web application. Endpoint risk detection techniques detect client-side vulnerabilities at the endpoint (the user). There are a good number of implementations, such as FLAX and Zozzle that target JavaScript issues. Another form of detection is called the middlebox risk detection. This kind of detection requires no adjustments of the code, as the detection is performed in between the server and client-side, using custom HTTP requests. Projects implementing this kind detection, such as SpyProxy, BrowserShield and WebShield, are still in early development stages, but they already look very promising.


-	Data security
-	Data locality
-	Data integrity 1,2
-	Data redundancy cancellation
-	Data access 1,2
-	Network security
-	Data confidentially issue
-	Availability
-	Backup

- Related Security issues
Taking a wider scope on the subject, papers indicate alot of security issues closely linked to multi-tenancy. These security issues should be taken into account due to the following two reasons.
First off, as mentioned earlier, the definition of multi-tenancy is still quite ambiguos. It is often used to indicate all sorts of cloud services, including IaaS and PaaS models, as seen by [...]. The scope of multi-tenancy is often larger than our definition of multi-tenancy, as illustrated by the paper [6]. Due to the increased scope, more security issues can be considered to belong to multi-tenancy. 
Another reason for taking into account related security issues, is the fact that multi-tenancy is high-level model [...]. The model depends on a bundle of underlying technologies, such as the hardware infrastructure, operating systems and server software. Each of these technologies has its own share of security issues, impacting the level of security of the multi-tenancy system on top. In the next segments we will explore related security issues, based upon a categorization by layer.

(
% Not all subjects can/should be treated
+ Virtual Machine security
-	Introspection
-	Redundancy Cancellation
-	VM-based Rootkits

+ Virtual Machine Monitor (VMM) security
-	VMM Vulnerabilities
-	VMM transparency
-	Platform Integrity

+ Platform security
-	Resource accounting
-	Safe thread termination
-	Confidentiality and integrity

+ OS layer security
-	Privilege separation
-	Kernel integrity
-	Prevention of execution of memory zones
)

- Future Work
As mentioned in previous sections the security aspect of Multi-Tenancy is relatively new. The survey of the literature regarding security, revealed the following recommendations for prospective researchers to look in to. [1] suggests research to be done on implementation specific security risks. In particular they suggest more research on methods on how to stop non-trusted threads without affecting the platform safely. Furthermore the paper proposes more research on mechanisms that allow resource sharing policies to be properly implemented. More research by [2] is needed in general on some of the technically immature areas, such as the security of the web-layer and hypervisor-layer. Those areas of security are relatively new and, although they have some challenging issues, haven't got any good, realistic solutions yet. [3] proposes an intensive analysis of the proposed authentication model. Furthermore, it looks for an analysis and comparison of other trust models and their suitability for cloud computing. According to [4] we need to have more research on more advanced trust models, next to having more experimentation with different database-systems for the proposed authorization system. [5] suggests more investigative research on some of the incomplete issues discussed in the paper. In [6] it is announced that the authors are looking into further improvement in areas such as authentication and the protection against VM attacks. To summarize, there is a lot research to be done on the security issues of multi-tenancy systems. Particularly, the papers in the survey aim for more research on practical implementations of the mostly theoretical solutions proposed.
