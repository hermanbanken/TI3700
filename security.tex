As multi-tenancy is a relatively new concept, the security aspect of multi-tenancy has not received much attention. The importance of security is particularly apparent in the adoption process of the multi-tenant model. 
Many potential businesses which would be interested in multi-tenancy are still reluctant to adopt it due to security to security and privacy concerns, according to Bernabe et al~\cite{Bernabe2012Auth}. 
Though multi-tenancy is targeted to provide better utilization of resources using virtualization techniques and to take up much of the work load from the client, it is fraught with security risks, following Seccombe et al~\cite{Seccombe2009Security}. 
According to Takahashi~\cite{Takahashi2012Security} this implies the design of strong security boundaries in order to isolate tenants when using these shared resources. 
Thus, to advance the development and deployment of multi-tenancy, security issues and approaches for the techniques range and cross various technical domains including cryptography, virtualization and programming, as noted by Takahashi~\cite{Takahashi2012Security}. \\

As mentioned in the previous sections, a single definition for multi-tenancy has yet to be established. 
These different definitions of the concept of multi-tenancy concern different systems and therefore different security concerns.
For example, the paper by Bezemer~\cite{bezemer2010multi} indicates that a multi-tenant system could be defined as having a single database for all tenants, or as having separate databases for each of tenants. 
Naturally, these systems will have a different number of security concerns to deal with. In this paper we mainly address issues related to true multi-tenancy as defined by Bezemer~\cite{bezemer2010multi}.

In this paper we will discuss the issues caused by or enlarged specifically by the use of multi-tenancy. 
Many other issues, such as platform security and virtualization, are concerns concerning the whole concept of cloud computing, not specifically multi-tenancy. These relevant issues are therefore not within the scope of this paper. 
However, some of the most relevant security issues will be discussed briefly, to give an indication of security concerns influencing the overall security of multi-tenancy.

\subsection{Security Issues with Multi-Tenancy}
In this section we will elaborate on the security aspect of the multi-tenancy model. 
We will discuss security issues inherently caused by the multi-tenancy model.
The security issues covered in this section are localization, secure data storage and authentication and authorization.
For each of these issues we will describe the security concern, followed by the current solutions for these issues. 
Lastly, we will list the remaining challenges in the research agenda.

\subsubsection{Localization}
One of the data confidentiality issues with multi-tenancy systems is the issue regarding the physical location of the data. 
Although it is easy to forget, all the data stored by the clients in a multi-tenant system still has to be stored somewhere on a physical location. 
The choice of the physical company can cause a lot of privacy and legal issues. 
According to Softlayer~\cite{Softlayer2009Security}, compliance and data privacy laws in various countries locality of data are of utmost importance in many enterprise architectures. 
For example, in many European and South American countries, certain types of data cannot leave the country because of potentially sensitive information, according to Subashini~\cite{Subashini2010Security}.
Besides these restrictions, there is the issue of jurisdiction.
The often difficult question is raised which locality has the jurisdiction over the data, when an investigation would occur.
As stated in~\cite{Subashini2011Security}, a secure model must be capable of providing reliability to the customer regarding on the location of the data of the customer.
However, the multi-tenancy concept makes the control of the physical location of the data more difficult. The data of all tenants is mixed, due to the shared database of the multi-tenancy concept, and is therefore hard to adapt to comply with local law.\\

Although localization is a huge security issue in multi-tenancy, by itself data localization is a field of research older than the concept of multi-tenancy, as illustrated by the paper of Johnson et al~\cite{Johnson1996Law}. 
The potential delivery and deployment models that attempt to make multi-tenant systems comply with the issue of localization are discussed by Mahmood~\cite{Mahmood2011Security}.
Other suggestions for improvements to ensure the privacy of data in specifically multi-tenancy are made by Chen et al~\cite{Chen2012Security}. These suggestions include a proposal for a Bayesian data distributing system, which distributes the data while attempting to uphold predefined constraints. For example, a tenant could specify that a specific set of data may only be kept in a few selected countries, to prevent breaking privacy laws in other countries. 

\subsubsection{Secure Data Storage}
One of the characteristics of multi-tenancy is that tenants are only able to view and modify their own data.
Data isolation is a difficult security issue for these systems, due to the fact that all tenants share the same application functionality and databases. 
Malicious tenants could use potential loop holes to hack their way to access of the data of other tenants. 
In contrast to regular cloud applications, multi-tenant systems are more at risk of data leakage, as tenants share a single database.
Tenants are often allowed to add custom code to these services, which makes the risk of data intrusion even bigger when precautions are not properly taken. 
A multi-tenant model should therefore ensure a clear ‘firewall’ for each tenant’s data.
The boundary must be ensured not only at physical level but additionally at the application level, as stated in the paper by Subashini~\cite{Subashini2011Security}.\\
%Kan nog wat over data redundancy cancellation

To ensure the secure data storing, the paper of Takahashi et al~\cite{Takahashi2012Security} suggests the use of encrypted data manipulation using crypto-graphical techniques. 
One such technique is called \acf{PECE}, which allows a user to encrypt a file in multiple layers, while being able to decrypt with a single key.
This technique would allow the tenant, the multi-tenant provider and any middleware providers to encrypt the data separately, while only allowing the tenant to hold the key for decryption.
Next to that, the authors propose the use of homomorphic encryption. 
This crypto-graphical technique enables the users to perform operations on encrypted files without the need to decrypt the files.
In a multi-tenant environment, this technique would increase the security for the tenants, because encrypted data would be decrypted and prone to leakage.
% Challenge: improve homomorphic encryption

\subsubsection{Authentication and Authorization}
The matter of authentication and authorization, often referred to as data access control, in multi-tenancy has been discussed extensively over the last years. 
Simple authentication schemes, where a user either has or has no access to all content, are widely available.
According to Bernabe et al~\cite{Bernabe2012Auth} current providers such as Rackspace\footnote{http://www.rackspace.com/} or Amazon EC2\footnote{http://aws.amazon.com/ec2/} only rely on simple authentication schemes which do not provide enhanced access control capabilities. 
Multi-user authentication provides enhanced access control capabilities.
However, this form of authentication lacks multi-tenancy-specific capabilities, while the user authorization is confined to each individual tenant. 
The multi-user authorization concept provides no capabilities to grant users privileges over multiple tenants, as noted by Calero et al~\cite{Calero2010Auth}
This would not allow, for example, two companies or tenants to share certain data with each other, by granting each other specific privileges as a form of collaboration.
These systems often lack the functionality and complexity to express more advanced forms of authorization, necessary for multi-tenant systems.
Thus, the problem lies with designing more advanced schemes of authentication, where users can be granted more detailed custom privileges.\\

Noteworthy progress can be noted in Bernabe~\cite{Bernabe2012Auth}, which proposes an access control model system suitable for multi-tenancy and grants high expressiveness in terms of permissions. 
Additionally, this expressiveness is supported by the integration of semantic web technologies into the authorization model. 
The system allows a fine-grained definition of what resources should available for each particular tenant. 
Another influence in this area is the system proposed in Calero~\cite{Calero2010Auth}. 
This authorization system is able to support collaboration agreements, often referred to as federations, between tenants or businesses.

\subsection{Related Security Issues}
Taking a wider scope on the subject, the surveyed papers indicate a lot of security issues closely linked to multi-tenancy. 
These security issues should be taken into account due to the following two reasons.
First off, as mentioned earlier, the definition of multi-tenancy is still quite ambiguous. 
It is often used to indicate all sorts of cloud services, including IaaS and PaaS models, as seen by Jasti et al~\cite{Jasti2010Security}. 
The scope of multi-tenancy is often larger than our definition of multi-tenancy. Due to the increased scope, more security issues can be considered to belong to multi-tenancy.
Another reason for taking into account related security issues, is the fact that multi-tenancy is a high-level model. 
The model depends on a bundle of underlying technologies, such as the hardware infrastructure, operating systems and server software. 
Each of these technologies has a particular share of security issues, impacting the level of security of the multi-tenancy system on top. \\


\subsubsection{\acf{VM} Security}
When taking virtual machines into account in multi-tenancy, introspection of virtual machines poses major challenges to the multi-tenant system. 
These systems implement the concept of a \acp{VMM}, which is tasked to manage and control the various \acp{VM}s needed by the tenants.
In these systems, a tenant generally has the ability to migrate his custom \acp{VM} to another \acp{VMM}. 
However, when a \ac{VM} is placed in a server with an untrusted \ac{VMM} it would allow \ac{VMM} to track the data flows inside the guest VM, as noted by Takahashi~\cite{Takahashi2012Security}.

Another current issue is the adaptation of VM-based Root-kits. 
These root-kits, in contrast to traditional root-kits, do not stop at OS level, but continue to attempt to infect the supervising \ac{VMM}. 
According to Takahashi~\cite{Takahashi2012Security}, several proof-of-concept VM-based root-kits, such as Blue Pill, were able to successfully identify and infect the VM, followed by the \ac{VMM}.

\subsubsection{Virtual Machine Monitor Security}
The \acl{VMM} has the essential role of isolating and controlling the virtual machines, which in multi-tenancy are managed by the tenants.
However, an investigation was conducted by Ormandy et al~\cite{Ormandy2007Security}of six major \acp{VMM} and emulators, using source code auditing techniques. All six systems had major flaws, leading to unexpected aborts and possible exploits.

Additionally there is the problem of the detectability of the \ac{VMM}’s. Ideally, the \ac{VMM} is completely transparent; the tenant has no notion what kind of \ac{VMM} is running the virtual machines. 
However, as argued in the paper by Takahashi~\cite{Takahashi2012Security}, the idea of complete \ac{VMM} transparency is unrealistic. 
Clues provided by the \ac{VMM}, such as time sources and overhead, which can be used to identify the type of \ac{VMM}
The detectability of the \ac{VMM} creates the opportunity for malicious users to target specific \ac{VMM} systems and versions.

\subsubsection{Web-dependent Application Security}
Web-dependent Application Security ensures the accessibility of the data. 
The importance of good security practices in the application-layer is be illustrated by Wade et al~\cite{Wade2008Security}.
The report about data breaches on the Verizon Business platform, reports that 39\% of the data breaches occur in the service/application layer, which comprises the multi-tenant section. The \ac{OWASP}\footnote{http://owasptop10.googlecode.com/files/OWASP\%20Top\%2010\%20-\%202013.pdf} has identified the 10 greatest security risks faced by network-dependent applications.\\

In Takahashi et al~\cite{Takahashi2012Security} the authors describe a couple of ways to detect vulnerabilities in the server-side and client-side of the web application. 
Endpoint risk detection techniques detect client-side vulnerabilities at the endpoint (the user). 
There are a good number of implementations, such as FLAX~\cite{saxena10kudzu} and Zozzle~\cite{curtsinger2011zozzle} that target JavaScript issues. 
Another form of detection is called the middle-box risk detection. 
This kind of detection requires no adjustments of the code, as the detection is performed in between the server and client-side, using custom HTTP requests. 
Projects implementing this kind detection, such as SpyProxy\footnote{\url{http://homes.cs.washington.edu/~gribble/papers/spyproxy.pdf}}, BrowserShield\footnote{\url{http://research.microsoft.com/en-us/news/features/browsershield.aspx}} and WebShield\footnote{\url{http://www.isoc.org/isoc/conferences/ndss/11/pdf/6_2.pdf}}, are still in early development stages, but these projects already look very promising.

\subsubsection{Data Integrity and Network Security}
With the dependence on extensive usage of networks, the multi-tenancy model is highly dependent on good network security.  
Multi-tenant systems have an extended emphasis on data integrity, because data transmissions of one tenant also need to be secure on the internal level of the multi-tenant system to prevent other tenants from accessing the potentially classified data.
According to Subashini~\cite{Subashini2011Security}, one of the biggest challenges with multi-tenant services is transaction management. 
At the protocol level, HTTP does not offer any support for transaction or guaranteed delivery of packets. 
Thus, to ensure transactions are indeed delivered one needs to implement this functionality into the multi-tenancy system.\\

Currently there are some standards available trying to fix this security issue, namely WS-Transaction\footnote{http://msdn.microsoft.com/en-us/library/ms951262.aspx} and WS-Reliability.
However, as noted by Subashini et al.~\cite{Subashini2011Security} and confirmed by our web-based survey based upon these techniques, these standards haven’t reached technical maturity yet and have therefore yet to experience full adoption by the majority of the multi-tenancy providers.
Since the publishing of the paper by Subashini.~\cite{Subashini2011Security}, WS-reliability has since been superseded by ReliableMessaging\footnote{http://docs.oasis-open.org/ws-rx/wsrm/200702/wsrm-1.1-spec-os-01.pdf}.

\subsection{Research Agenda for Security}\label{sec:security_agenda}
The survey of the literature regarding security revealed the following recommendations for researchers to look in to. 
\begin{itemize}
    \item \textbf{Analysis of proposed authentication models}.
        The papers of Bernabe~\cite{Bernabe2012Auth} and Calero~\cite{Calero2010Auth} proposes an intensive analysis of the proposed authentication models. Furthermore, there needs to be more research on more advanced authorization models, next to having more experimentation with different database-systems for the proposed authorization system.
    \item \textbf{Tradeoff between security and performance}.
        Although a lot of security measures are proposed to secure multi-tenant systems, more research should dedicated to finding a balance between security and performance, according to Guo~\cite{guo2007framework} and Hashizume~\cite{Hashizume2013Security}. Traditional and new Security mechanisms should be redesigned to increase the effectiveness of the mechanisms in multi-tenancy environments.
    \item \textbf{Improve data security}.
        More research needs to be conducted on techniques to ensure that data of tenants is completely isolated. Currently, many papers, such as Jasti~\cite{Jasti2010Security}, Merino~\cite{Merino2011Security} and Takahashi~\cite{Takahashi2012Security}, have pointed out that there are many methods on different levels, ranging from VM security to data localization, to comprise confidential data. 
\end{itemize}

