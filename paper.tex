\documentclass{article}
\usepackage{a4wide}
\usepackage{graphicx}
\usepackage{float}
\usepackage{url}
\usepackage{fancyhdr}
\usepackage{geometry}
\usepackage{acronym}
\usepackage{lastpage}
\usepackage{color}
\newcommand{\highlight}[1]{\colorbox{yellow}{#1}}
\urlstyle{same}

\acrodef{MTA}{Multi-Tenant Application}
\acrodef{GAE}{Google App Engine}
\acrodef{VP}{variation point}
\acrodef{AOSD}{Aspect Oriented Software Development}
\acrodef{VRT}{Variation Realization Technique}
\acrodef{COP}{Context Oriented Programming}
\acrodef{DI}{Dependency Injection}
\acrodef{DaaS}{Data as a Service}
\acrodef{IaaS}{Infrastructure as a Service}
\acrodef{PaaS}{Platform as a Service}
\acrodef{SaaS}{Software as a Service}
\acrodef{OVM}{Orthogonal Variability Model}
\acrodef{SLA}{Service Level Agreement}
\acrodef{MVC}{Model View Controller}
\acrodef{QoS}{Quality of Service}
\acrodef{VM}{Virtual Machine}
\acrodef{DBMS}{Database Management System}
\acrodef{RDBMS}{Relational Database Management System}
\acrodef{VMM}{Virtual Machine Monitor}
\acrodef{PECE}{Progressive Elliptic Curve Encryption}
\acrodef{OWASP}{Open Web Application Security Project}
\acrodef{FUP}{Fair Use Policy}
\acrodef{AWS}{Amazon Web Services}

\title{Multi-Tenancy: ready to rock? a survey}
% We moeten nog wat leukers verzinnen...
\author{Herman Banken\and
    Jasper Dijt\and
    Erwin van Eyk\and
    Rick Wieman}
\date{\today}

\pagestyle{fancy}
\lhead{Multi-Tenancy: ready to rock? a survey}
\rhead{page \thepage\ of \pageref{LastPage}}
\cfoot{}

\begin{document}
\maketitle
\thispagestyle{empty}

\begin{abstract}
Multi-tenancy allows multiple organizations to use a single application with their own configuration, on the same system. Despite many authors have researched multi-tenancy, no clear definition exists.

In our survey, we research the current state of multi-tenancy. We identified four major software concepts that need extra care or provide extra opportunities in multi-tenant application development: 
security, as the security concerns are holding back adoption of multi-tenancy; 
scalability, as an unscalable application cannot leverage the benefits of the economics of scale; 
quality of service, as performance must be upheld for all active tenants;
and variability, as that's where value is added for tenants.  

Our Research~Agenda provides an overview of possible research opportunities, 
like... \highlight{een paar highlights}% een paar highlights
\\

\textbf{Keywords}: multi-tenancy, security, scalability, quality of service, variability
\end{abstract}

\section{Introduction}
Due to the uprising of cloud services, more and more applications are converted into multi-tenancy applications. 
This allows multiple organizations with multiple users to use the same application, and allows application providers to achieve better economies of scale. 
There is however not an explicit relationship between multi-tenancy and the cloud, since there are single-tenant cloud applications and vice versa. 
Additionally, multi-tenant applications pose new challenges, for example in terms of security and scalability. 
This literature survey merges the information of about 50 papers, in order to create a clear view of the definitions of multi-tenancy, the challenges and the research opportunities.

This survey continues with some background information on multi-tenancy in Section~\ref{sec:bg}. Thereafter, the most common challenges for multi-tenancy will be discussed: security (Section~\ref{sec:security}), scalability (Section~\ref{sec:scalability}), \ac{QoS} (Section~\ref{sec:qos}) and variability (Section~\ref{sec:variability}). After these challenges, we discuss the research opportunities in Section~\ref{sec:ra} and we conclude the survey in Section~\ref{sec:conclusion}. % Moet hier nog een waarom bij? Nee toch?

\section{Background information}
\label{sec:bg}
Multi-tenancy is a vague term, since it has never had an exact and official definition. During the last years, also due to the uprising of cloud applications, multi-tenancy gets researched and used more and more. In this section we give some recent definitions and we shortly describe the various types of multi-tenancy. We also discuss the relationship with the cloud and mention the challenges of multi-tenancy.

\subsection{Definitions}

In multi-tenant research, there are three important concepts: tenants, multi-tenancy, multi-user. In this section we show the variety of definitions of these concepts.

Bezemer and Zaidman~\cite{bezemer2010multi} define a tenant as an organizational entity which rents a multi-tenant \ac{SaaS} solution (usually grouping a number of users). Krebs et al.~\cite{krebs2012architecture} have a more loose definition of a tenant; they define it as a group of users sharing the same view.

Based on these definitions, both more or less agree on the definition of a multi-tenant application: hardware resource sharing by offering one shared application and database instance to multiple tenants.

The difference between multi-tenancy and multi-user is also made clear by Bezemer and Zaidman~\cite{bezemer2010multi}. They state that in a multi-user application all users use the same application with limited configuration options, whereas a multi-tenancy application has more configuration options. However, all users use the same application, in both multi-user and multi-tenancy applications.

\subsection{Different types of multi-tenancy}

Krebs et al.~\cite{krebs2012architecture} describe various layers of sharing, in order of increasing benefits:
\begin{enumerate}
\item Sharing a data center (very limited benefits)
\item Virtualization, thus sharing a server (large overhead)
\item Middleware sharing (difficult isolation and overhead)
\item Multi-Tenant Application
\end{enumerate}

According to Bezemer and Zaidman~\cite{bezemer2010multi}, there are also various ways of implementing multi-tenancy: a shared application with a separate database, a shared application with a shared database with a separate table and a shared application, with a shared table. They considere the latter as pure multi-tenancy. Krebs et al.~\cite{krebs2012architecture} divide this into affinity and persistency, with affinity describing which server(s) handle which tenant(s) and with persistency describing the way of usage of the database (shared or separated).

Chang Jie Guo et al.~\cite{guo2007framework} suggest the implementation of a multi-tenancy enablement layer. This should create a separation between usage and resources, creating the required isolations and allowing customizations. This can be compared to the blueprint of Bezemer and Zaidman~\cite{bezemer2010multi}, since both approaches should have little impact on the single-tenant code.

\subsection{Multi-tenancy vs. the cloud}

Dillon et al.~\cite{dillon2010cloud} comprehensively elaborate on various aspects of cloud services. They mention the following service models: 
\begin{enumerate}
\item \ac{SaaS} (example: Google Mail/Docs)
\item \ac{PaaS} (example: Google AppEngine)
\item \ac{IaaS} (example: Amazon EC2)
\item \ac{DaaS} (example: Google BigTable)
\end{enumerate}

Multi-tenancy falls mostly within the \ac{SaaS} domain, as mentioned by Tsai et al.~\cite{tsai2010towards}. Multi-tenancy gains the most benefits within this model.

Dillon et al.~\cite{dillon2010cloud} also describe the several essential characteristics of cloud services, including resource pooling ('pooling' computing resources together in an effort to serve multiple consumers) and rapid elasticity (the consumption of resources can rapidly increase and the usage is not predictable upfront). These are both challenges for multi-tenant applications, as also mentioned by Krebs et al.~\cite{krebs2012architecture} and Bezemer and Zaidman~\cite{bezemer2010multi}.

\subsection{Challenges of multi-tenancy}

The multi-tenancy model has created two new security issues~\cite{dillon2010cloud}. Sharing resources on the same physical machine pose a danger to the data of the tenants. This falls within the isolation challenge. Another issue is reputation fate-sharing, since one might be sharing resources with possible criminal users, creating the possibility to (for example) get blacklisted.

Another challenge is the charging model. The costs of developing multi-tenancy can be very substantial for \ac{SaaS} providers, because they need to re-design or re-develop single-tenant software, introduce new features for customization and enhance the security.

Bezemer and Zaidman~\cite{bezemer2010multi} also mention performance as a challenge, since multiple tenants are using the same hardware resources. Other challenges are scalability (usage of resources can suddenly increase), zero-downtime (the user expects the system to be online when he needs it) and maintenance (added complexity due to multi-tenancy could make code harder to maintain).

\section{Security}
\label{sec:security}
\subsection{Security in Multi-Tenancy}
Like the concept of multi-tenancy itself, the security aspect of multi-tenancy is relatively new and open to more research. As discussed in other sections, the model offers several significant advantages in comparison to other software models. While the cloud offers advantages, until some of the risks are better understood, many of the major players will be tempted to hold back. To advance the development and deployment of cloud computing, security issues and approaches for the techniques range and cross various technical domains including cryptography, virtualization and programming \cite{Takahashi2012Security}.
% Could be left out... ^
% These two sections could be merged..

\subsection{Importance of Security}
The importance of decent security is particularly apparent in the adoption process of the multi-tenant model. Many potential businesses which would be interested in cloud computing are still a bit reluctant to adopt it due to security to security and privacy concerns, according to Bernabe et all\cite{Bernabe2012Auth}. The concept of multi-tenancy consists of having tenants share IT infrastructure and the sharing of resources. Though cloud computing is targeted to provide better utilization of resources using virtualization techniques and to take up much of the work load from the client, it is fraught with security risks (Seccombe et al, 2009). According to Takahashi\cite{Takahashi2012Security} this implies the design of strong security boundaries in order to isolate tenants when using these shared resources. Thus, the system to control the access to the available resources becomes a critical aspect in order to provide an efficient control over the usage of the cloud architecture. \\
Next to technical security issues, privacy questions also need to be addressed to realize the potential of this new model. Cloud computing moves the application software and databases to the large data centers, where management of the data and services are not trustworthy. This unique attribute poses many new security challenges (Cong Wang et al, 2009). 

\subsection{Security Issues with Multi-Tenancy}
In the following sections we will expand upon the security aspect in the multi-tenancy model. First of all, in first sections we will discuss current security issues in the multi-tenancy model. \\
Secondly, in the following sections some of the security issues closely linked to multi-tenancy, such as on the aspect of virtualization, will be explained. Finally we list the remaining challenges and future work in the final section.

\subsubsection{Localization}
One of the data confidentiality issues with multi-tenancy systems is the issue regarding the physical location of the data. Although it is easy to forget, all the data stored by the clients in the cloud still has to be stored somewhere on a physical location. The choice of the physical company can cause a lot of privacy and legal issues. According to (Softlayer, 2009), compliance and data privacy laws in various countries locality of data are of utmost importance in many enterprise architectures. For example, in many European and South American countries, certain types of data cannot leave the country because of potentially sensitive information. Besides these restrictions, there is also the issue of jurisdiction. The often difficult question is raised which locality has the jurisdiction over the data, when an investigation would occur. As stated in \cite{Subashini2010Security}, a secure model must be capable of providing reliability to the customer regarding on the location of the data of the customer.

\subsubsection{Secure Data Storage}
One of the most important characteristics of a good multi-tenant system is that tenants are only able to view and modify their own data. This is a relatively difficult security issue for these systems, due to the fact that all tenants share the same application functionality and databases. Malicious tenants could try to use loop holes to hack their way to access of the data of other tenants. Tenants are often allowed to add custom code to these services, which makes the risk of data intrusion even bigger when precautions aren’t properly taken. A multi-tenant model should therefore ensure a clear ‘firewall’ for each tenant’s data. The boundary must be ensured not only at physical level but also at the application level, as stated in the paper by Subashini \cite{Subashini2010Security}. \\
%[Kan nog wat over data redundancy cancellation and backup)
To ensure the secure data storing, the paper \cite{Takahashi2012Security} suggests the use of encrypted data manipulation using cryptographical techniques. One such technique is called progressive elliptic curve encryption (PECE), which allows a user to encrypt a file with multiple keys, while decrypting it with a single key. Next to that, they also propose Homomorphic encryption. This cryptographical technique enables the users to perform operations on encrypted files without decrypting the value.

\subsubsection{Authentication and Authorization}
The matter of authentication and authorization, also often referred to as data access control, in multi-tenancy has been discussed extensively over the last years. Simple authentication schemes, where a user either has or has no access to all content, are widely available. According to Bernabe et all \cite{Bernabe2012Auth} current cloud providers such as Rackspace \footnote{http://www.rackspace.com/} or Amazon EC2 \footnote{http://aws.amazon.com/ec2/} only rely on simple authentication schemes which do not provide enhanced access control capabilities. These systems often lack the functionality and complexity to express more advanced forms of authorization. Thus, the problem lies with designing more advanced schemes of authentication, where users can be granted custom privileges, besides having full administrator access.\\
Noteworthy progress can be noted in bernabe\cite{Bernabe2012Auth}, which proposes an access control model system suitable for multi-tenancy and grants high expressiveness in terms of permissions. This expressiveness is also supported by the integration of semantic web technologies into the authorization model. The system allows a fine-grained definition of what resources should available for each particular tenant. Another influence in this area is the system proposed in Calero\cite{Calero2010Auth}. This authorization system is able to support collaboration agreements, often referred to as federations, between tenants or businesses.

\subsubsection{Web Application Security}
The web is an inevitable aspect of the multi-tenancy model. It ensures the accessibility of the data. The importance of good security practices in the application-layer can be illustrated by (Wade et al, 2008). The report about data breaches on the Verizon Business platform, where the following:\\
\begin{enumerate}
    \item Application/service layer – 39\%
    \item OS/platform layer – 23\%
    \item Exploit known vulnerability – 18\%
    \item Exploit unknown vulnerability – 5\%
    \item Use of back door – 5\%
\end{enumerate}
The statistics are quite clear about it; nowadays hackers and other malicious users tend to attack the higher layers, like the application-layer. The Open Web Application Security Project (OWASP) \footnote{http://owasptop10.googlecode.com/files/OWASP\%20Top\%2010\%20-\%202013.pdf} has identified the 10 greatest security risks faced by web applications, including SQL injections and cross-server scripting. \\
In Takahashi et all\cite{Takahashi2012Security} the authors describe a couple of ways to detect vulnerabilities in the server-side and client-side of the web application. Endpoint risk detection techniques detect client-side vulnerabilities at the endpoint (the user). There are a good number of implementations, such as FLAX and Zozzle that target JavaScript issues. Another form of detection is called the middle-box risk detection. This kind of detection requires no adjustments of the code, as the detection is performed in between the server and client-side, using custom HTTP requests. Projects implementing this kind detection, such as SpyProxy, BrowserShield and WebShield, are still in early development stages, but they already look very promising.

\subsubsection{Data Integrity and Network Security}
With the dependence on extensive usage of networks, the multi-tenancy model is also highly dependent on good network security.  The data transmissions over the network needs to be secured to prevent loss or theft of classified information. According to Subashini\cite{Subashini2011Security}, one of the biggest challenges with multi-tenant services is transaction management. At the protocol level, HTTP does not offer any support for transaction or guaranteed delivery of packets. Thus, to ensure transactions are indeed delivered one needs to implement this functionality into the multi-tenancy system.\\
	Currently there are some standards available trying to fix this security issue, namely WS-Transaction \footnote{http://msdn.microsoft.com/en-us/library/ms951262.aspx} and WS-Reliability, which has been superseded by ReliableMessaging \footnote{http://docs.oasis-open.org/ws-rx/wsrm/200702/wsrm-1.1-spec-os-01.pdf}. However, noted by (subashini, 2011), these standards haven’t reached technical maturity yet and have therefore yet to experience full adoption by the majority of the multi-tenancy providers.

\subsection{Related Security Issues}
Taking a wider scope on the subject, papers indicate a lot of security issues closely linked to multi-tenancy. These security issues should be taken into account due to the following two reasons.\\
First off, as mentioned earlier, the definition of multi-tenancy is still quite ambiguous. It is often used to indicate all sorts of cloud services, including IaaS and PaaS models, as seen by Jasti et all\cite{Jasti2010Security}. The scope of multi-tenancy is often larger than our definition of multi-tenancy. Due to the increased scope, more security issues can be considered to belong to multi-tenancy. \\
Another reason for taking into account related security issues, is the fact that multi-tenancy is a high-level model. The model depends on a bundle of underlying technologies, such as the hardware infrastructure, operating systems and server software. Each of these technologies has its own share of security issues, impacting the level of security of the multi-tenancy system on top. 

\subsubsection{Virtual Machine Security}
When taking virtual machines into account in multi-tenancy. Introspection of virtual machines pose major challenges to the cloud provider. A tenant has the ability to migrate his custom VMs to other VMMs. However, when a VM is placed in a server with an untrusted VMM it would allow VMM to track to the data flows inside the guest VM \cite{Takahashi2012Security}. \\
Another current issue is the adaptation of VM-based Root-kits. These root-kits, unlike traditional root-kits, do not stop at OS level, but continue to attempt to infect the supervising VMM. According to Takahashi\cite{Takahashi2012Security}, several proof-of-concept VM-based root-kits, such as Blue Pill, were able to successfully identify and infect the VM, followed by the VMM.

\subsubsection{Virtual Machine Monitor Security}
The Virtual Machine Monitor (VMM), often referred to as the hypervisor, has the essential role of isolating and controlling the virtual machines. However, an investigation was conducted by Ormandy et all\cite{Ormandy2007}of six major VMMs and emulators, using source code auditing and fuzzing techniques. All six systems had major flaws, leading to unexpected aborts and possible exploits.\\
Additionally there is the problem of the detectability of the VMM’s. Ideally, the VMM is completely transparent; the tenant has no notion what kind of VMM is running the virtual machines. However, as argued in the paper by Takahashi\cite{Takahashi2012Security}, that the idea of VMM transparency is unrealistic, due to a significant number of implicit clues, such as time sources and overhead, given by the system. The detectability of the VMM creates the opportunity for malicious users to target specific VMM systems and versions.

\subsubsection{Platform Security}
Major platforms, such as the Java platform, also contain certain limitations that hinder the construction of secure multi-tenant systems. Resource accounting is one of those relevant issue on platforms. The security around the usage of resources is, as explained in other issues, rather simplified. Once access is granted to some code, that code can use the resource without limitations, as argued by Merino\cite{Merino2011Security}. There is no accounting of resource usage by threads. A hostile user could try to create memory-overflows by creating many instances of objects.
An issue present in the major platforms is safe thread termination. This problem is due to the lack of a safe way to enforce thread termination and is present in Java and .NET platforms \cite{Merino2011Security}. Aborting a thread in Java and .NET languages can be thought of as a strongly worded suggestion to the thread that it should stop functioning; a thread running untrusted code can just ignore it and continue executing possibly hostile code.

\subsubsection{OS Layer Security}
Like all the stacked layers of software, integrity of the OS is a major concern. As stated in the paper by Takahashi\cite{Takahashi2012Security} the chain of trust should be assured at every layer of the software stack. With the execution of possible untrusted software on top of the OS, creates the possibility for hackers to exploit known vulnerabilities of the operating system to gain access to the complete system, when security preventing these practices is lacking.
% Could be left out; not very interesting...


\subsection{Future Work on Security}
As mentioned in previous sections the security aspect of Multi-Tenancy is relatively new. The survey of the literature regarding security, revealed the following recommendations for prospective researchers to look in to. \cite{Moreno2011Security} suggests research to be done on implementation specific security risks. In particular they suggest more research on methods on how to stop non-trusted threads without affecting the platform safely. Furthermore the paper proposes more research on mechanisms that allow resource sharing policies to be properly implemented. More research by Takahashi\cite{Takahashi2012Security} is needed in general on some of the technically immature areas, such as the security of the web-layer and hypervisor-layer. Those areas of security are relatively new and, although they have some challenging issues, haven't got any good, realistic solutions yet. Bernabe\cite{Bernabe2012Auth} proposes an intensive analysis of the proposed authentication model. Furthermore, it looks for an analysis and comparison of other trust models and their suitability for cloud computing. According to Calero \cite{Calero2010Auth} we need to have more research on more advanced authorization models, next to having more experimentation with different database-systems for the proposed authorization system. Hashizume et all\cite{Hashizume2013Security} suggests more investigative research on some of the incomplete issues discussed in the paper. In [6] it is announced that the authors are looking into further improvement in areas such as authentication and the protection against VM attacks. To summarize, there is a lot research to be done on the security issues of multi-tenancy systems. Particularly, the papers in the survey aim for more research on practical implementations of the mostly theoretical solutions proposed.
% future work of 3 papers missing.


\section{Scalability}
\label{sec:scalability}
%INTRO: (section title is in main file.)
%\subsection{scalability.tex}
The scalability of a system, in general, describes how well a system can handle increasing workload, stored data and utilisation. 
A system is considered scalable when it can be easily extended to support an increasing load. 
For an unscalable system supporting increased load might be technically infeasable or prohibitively expensive.\cite{bondi2000scalability}

In the scalability section we will focus specifically on how a multi-tenant application can deal with an increasing number of tenants, increasing amount of data and an increasing amount of data variability. 
Seperation of tenants and maintaining performance during peak usage will be discussed in the QoS section.

%WERKTITELS!
\subsubsection{Where and how Scalability Matters}
As one of the key advantages of multi-tenant applications is a higher utilisation of resources and the associated cost reduction per tenant.\cite{bezemer2010multi} 
It is required that a multi-tenant application is scalable to maintain this cost reduction and to accomodate for the, hopefully, ever increasing number of tenants.
This also means that it is both undesirable to underprovision resources and to overprovision resources. The former causing performance degradation, the latter causing increased and unneeded costs.

%TODO: 	Uitleg research richtingen, plaatsing van tenants, hoe/WANNEER moet ik opschalen als er meer tenants bij komen
%	  	Uitleg DB onderzoeken. 

\subsubsection{Scalability in the data layer}
A database for a multi-tenant application should be extendable bij the tenants.
This allows for tenant specific modifications to the schema.
%TODO: Kijken in hoeverre dit al onder het algemene deel behandeld is.

Thus, scalability in the data layer is important in two different ways.
First, as the number of tenants increases so will the amount of data and load on the database.
Second, as the number of tenants increases so will the amount of customisations to the schema. 
The database schema should be managing this as effici\"ently as possible.

In the sections after this we will look at various schema mapping techniques and their pro's \ cons.
After we will discuss two prototype tenant aware DBMSes.

\subsubsection{Schema mapping techniques}
\begin{description}
	\item[Universal table(s)]
	\item[Private tables]
	\item[Sparse columns]
	\item[Extension field]
	\item[Extension tables]
	\item[Chunk folding] %Includes chunk tables! 
	\item[Pivot table]
\end{description}

\subsubsection{Tenant aware DMBSes}
%FLEXSCHEME and the tenant aware RDBMS.

\subsubsection{Resource allocation for new tenants}
%Allocation of resources for new tenants, predicting & preventing conflicts / bottlenecks.

\subsubsection{Discussion of challenges \& future work}
%Spreekt voor zich.


\section{Quality of Service}
\label{sec:qos}
In this section various methods for maintianing a good quality of service level per tenants will be discussed.
The main challenges here are the metrics to use for measuring the current resource usage per tenant and how to seperate tenants in such a way that the behaviour of one tenant will not affect the other tenants.\cite{krebs2013metrics}

%WERKTITELS
\subsubsection{Why QoS Matters}
Most tenants in a multi-tenant application will have some form of Service Level Agreement (SLA) with the provider of the application.
In many cases these SLA's contain certain Quality of Service constraints the provider will have to meet, such as response time or minimum uptime. 
The tenant on the other hand will also have certian limits and/or quotas, instead of unlimited access to the application.

Based on these SLA's the available resources should be fairly distributed among tenants.
Rouven Krebs\cite{krebs2013metrics} defines a system as fair when the following criteria are met:
\begin{enumerate}
	\item Tenants who do not exceed their quota should not be affected by tenats who do.
	\item Tenants who do exceed their quota should experience performance degradation. 
		It is important to note here that even if the system is not low on resources a tenant should still get degraded performance for the system to be fair.
		This is because otherwise the tenant would be getting more than is agreed upon in their SLA, wich is usually more than they are paying for and more than tenants under the same SLA who are not exceeding their quota.
	\item Tenants with a higher quote should experience better performance compared to tenants with a lower quota.
\end{enumerate}
% SLA's, eerlijk gebruik resources.

\subsubsection{Metrics for Tenant Load}
%ZIE: QoS 5.: Krebs et.al.  QoS 2.: Kwok et.al.

\subsubsection{Tenant Seperation Techniques}
%TODO: Let op: in hoeverre verschillende architectuurkeuzes al behandeld in het algemene deel.
% 1. VMs (tenants toekennen aan machines)
% 2. Tenants toekennen aan specifieke resources (thread pools e.d.) 
% 3. Alle tenants delen dezelfde resources (QoS via: request throttling / queueing.)

\subsubsection{Current Challenges in QoS Research}


\section{Variability}
\label{sec:variability}
In multi-tenancy \textit{variability} is a key concept. The term was first introduced in the car industry, where customers could choose certain \textit{variants} of chassis, engine and color \cite[p. 153]{kabbedijk2011variability}. 
In research on software engineering the concept was defined as ``the ability of a software system or artefact to be efficiently extended, changed, customized or configured for use in a particular context'' \cite{svahnberg2005taxonomy}.
Two keywords from this definition are customization and configuration. In a multi-tenant context configuration is preferred over customization \cite{sun2008software} as customization defines the process of reengineering an application, maintaining multiple branches and deploying these branches separately, while configuration can be done at run-time and does not require multiple instances or branches.

\subsubsection{Why is variability needed}
There are several reasons why variability is needed. 
First of all based on country, segment or branch different currencies, legislation and tax rules may apply. This is especially important in financial applications. 
Secondly different customers can require different functionality properties, layout options and/or quality of service (such as privacy and performance).

\subsubsection{Levels of variability}
Variability can be accomplished on different levels. 
Dependent on the scale of the application different patterns become be relevant. Large applications often consist of \textit{components} offering a specific \textit{service} and variability can be accomplished by dynamically swapping these components for different tenants, or by activating or deactivating specific components \cite{mietzner2008defining}. 
In smaller applications and within components different patterns become relevant. For example one can customize an application by using dependency injection \cite{walraven2011middleware} or context oriented programming \cite{truyen2012context}.

\subsubsection{variability techniques}
\subsubsection{variability modeling}
\subsubsection{future work in variability}

\section{Research Agenda}
\label{sec:ra}
The multi-tenancy model has created two new security issues~\cite{dillon2010cloud}. Sharing resources on the same physical machine pose a danger to the data of the tenants. This falls within the isolation challenge. Another issue is reputation fate-sharing, since one might be sharing resources with possible criminal users, creating the possibility to (for example) get blacklisted.

Another challenge is the charging model. The costs of developing multi-tenancy can be very substantial for \ac{SaaS} providers, because they need to re-design or re-develop single-tenant software, introduce new features for customization and enhance the security. % voordeel noemen (Guo 2007); citaties!

Bezemer and Zaidman~\cite{bezemer2010multi} also mention performance as a challenge, since multiple tenants are using the same hardware resources. Other challenges are scalability (usage of resources can suddenly increase), zero-downtime (the user expects the system to be online when he needs it) and maintenance (added complexity due to multi-tenancy could make code harder to maintain).

\section{Conclusion}
\label{sec:conclusion}
In this paper we surveyed the current state of multi-tenancy research.
We have identified several topics for further research on the subjects of security, scalability, quality of service and variability.

\highlight{Korte samenvatting hele paper}\\
Multi-tenancy can be divided in three levels: a shared application with a separated database, a shared application with a shared database (separate schema) and a shared application with a shared schema~\cite{bezemer2010multi}. The latter is considered pure or native multi-tenancy~\cite{bezemer2010multi,lin2009feedback,aulbach2009comparison}.

The relationship between multi-tenancy and the \ac{SaaS} domain is considered close~\cite{dillon2010cloud}, or even stronger: multi-tenancy is a characteristic of the \ac{SaaS} domain~\cite{tsai2010towards}. Additionally, the challenges of cloud services overlap with the challenges of multi-tenancy~\cite{dillon2010cloud,krebs2012architecture}.

In security, a overview  has been provided of the new security concerns and their solutions introduced in multi-tenancy, including data localization, data storage and authentication and authorization. We showed that there is a thin line between security issues caused by multi-tenancy and security issues caused by the general cloud technologies. Multi-tenancy is a high-level concept, relying on a multitude of other technologies, which thereby inherits traditional security issues.  

In Scalability research on estimating resource consumption per tenant and using these estimations to place tenants within the existing infrastructure was discussed. In addition to that an overview of data layer specific scalability research was presented.

\ac{QoS} research was surveyed seperately from scalability. Three techniques that use a form of per tenant request limiting for seperating tenants where discussed.

In Variability, most research is done on modeling variations and describing techniques to build variants. We provided an overview of the levels of variability (e.g. external variability, visible to the customer; and internal variability, invisible to the customer), three modeling techniques (Feature modeling, Decision modeling and Orthogonal Variability modeling) and the various variation techniques.

\highlight{Belangrijkste research agenda punten}
\begin{itemize}

\item \textbf{Automation}. 
To exploit economies of scale, providers of multi-tenant applications want to attract as many tenants and users as possible. 
However managing all these tenants and their configuration, takes a lot of time without the proper tools. 
Research has been done on creating wizards for tenants~\cite{mietzner2008generation,mietzner2008defining}, but this field is not completely covered yet and additionally, automatic deployment of the outcome of these wizards can be researched. 
\item \textbf{Guarantees}. 
In all aspects of multi-tenancy new problems arise. 
The solutions proposed for these problems need to be provable, for multi-tenancy to be usable in high availability or business critical applications. 
Open research opportunities lie in state consistency while updating the platform (Section~\ref{sec:var_agenda}), proving data isolation between tenants (Section ~\ref{sec:security_agenda}), providing performance isolation and minimal performance (Section~\ref{sec:qos_agenda}).
\item \textbf{Effective security}.
Although security measures have been proposed to secure multi-tenant systems, research should dedicated to finding a effective balance between security and performance. Traditional and new Security mechanisms should be redesigned to increase their effectiveness in multi-tenancy environments.
\item \textbf{Tenant aware components}.
In scability and \ac{QoS} research tools like tenant aware load balancers and tenant aware databases are used or proposed.
The benefits of such tools in a multi-tenant environment have already been proven.
However in both the case of the multi-tenant aware database and multi-tenant aware load balancers a lot of work remains to be done.

\item \highlight{Voeg hier je belangrijkste punten toe}  en probeer te kijken of het niet overlapt met andere punten en zo ja: voeg samen.
\end{itemize}

\bibliographystyle{plain}
\bibliography{ref}

\end{document}
