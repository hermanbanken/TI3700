In multi-tenancy \textit{variability} is a key concept. The term was first introduced in the car industry, where customers could choose certain \textit{variants} of chassis, engine and color \cite[p. 153]{kabbedijk2011variability}. 
In research on software engineering the concept was defined as ``the ability of a software system or artefact to be efficiently extended, changed, customized or configured for use in a particular context'' \cite{svahnberg2005taxonomy}.
Two keywords from this definition are customization and configuration. In a multi-tenant context configuration is preferred over customization \cite{sun2008software} as customization defines the process of reengineering an application, maintaining multiple branches and deploying these branches separately, while configuration can be done at run-time and does not require multiple instances or branches.

\subsubsection{Why is variability needed}
There are several reasons why variability is needed. 
First of all based on country, segment or branch different currencies, legislation and tax rules may apply. This is especially important in financial applications. 
Secondly different customers can require different functionality properties, layout options and/or quality of service (such as privacy and performance).

\subsubsection{Levels of variability}
Variability can be accomplished on different levels. 
Dependent on the scale of the application different patterns become be relevant. Large applications often consist of \textit{components} offering a specific \textit{service} and variability can be accomplished by dynamically swapping these components for different tenants, or by activating or deactivating specific components \cite{mietzner2008defining}. 

In smaller applications and within components different patterns become relevant. For example, one can customize an application by using dependency injection \cite{walraven2011middleware} or context oriented programming \cite{truyen2012context}.

%reasons for variability [5]
%different papers describe different categorizations

\subsubsection{variability modeling}
% On a larger scale \textit{variation modeling} becomes important \cite{mietzner2009variability, schroeter2012towards} as it becomes harder to keep track of different \textit{variation points}. 

% from Customization Realization in Multi-tenant Web Applications, Jansen en Houben (2010)
%``Mietzner et al. [1,8] propose a variability modeling technique for multi-tenant sys- tems. The variability modeling technique is also rooted in the software product line research area, and defines internal (developer view) and external (customer view) vari- ability. Their variability modeling technique enables developers to speculate and calculate costs for unique instances within a multi-tenant deployment. Furthermore, Mietzner also addresses different deployment scenarios for different degrees of tenant variability. Unfortunately, they do not address the CRTs.''

% ... extend the concept of product variability by making a distinction between functionality required by customers and variability that is a consequence of (multiple) other variations.

\subsubsection{Variability Realization Techniques}
Jansen et al. \cite{jansen2010customization} note that while lot of research has been done on variability modeling, there is lack of research on techniques needed to implement variability. They identify five realization techniques of two types of customization, being \textit{Model View Controller (MVC) customization} and \textit{system customization}:
\begin{itemize}
\item \textbf{Model changes} allow tenants to fit the application to their domain. Model changes include changing entity attributes, adding entities and entity-relationships. Some changes require a more abstract application and therefore need to be thought upon earlier in the design process than more simple changes which could easily be added later on.
\item Some of the simpler customizations include \textbf{view changes} as in MVC patterns views don't interact with other components, but instead are the receiving part.
\item There is much freedom in \textbf{controller changes}, ranging from restricting data to one tenant only till fully customized workflows.
\end{itemize}
Large systems build of components delivering services can create variations of a different magnitude, on system level:
\begin{itemize}
\item \textbf{System connector changes} occur when connections to remote services are made variable. Examples include having multiple connectors for printing photos or multiple physical mail delivery providers.
\item With \textbf{system component changes} multiple components deliver the same service and are selected based on configuration.
\end{itemize}

Kabbedijk and Jansen \cite{kabbedijk2011variability} describe three variability realization techniques they observed in case studies. 
\textbf{Customizable Data Views} are a pattern to store tenant specific representation settings, eg. how tenants want to filter or sort their data. 
Although this technique also requires some table storing data and needs a small controller, it can be seen as mainly a view change as described by Jansen et al. 
Another view change is the \textbf{Module Dependent Menu}-pattern, to create menus that are dependent on the modules associated to a tenant. The last technique they observe is more a controller change. 
\textbf{Pre/Post Update Hooks} is a pattern to execute pieces of code before or after certain events happen in the application. 
This allows for tenants to enable or disable modules that check data or automate certain tasks.

Truyen et al. \cite{truyen2012context} introduce \textit{Context Oriented Programming (COP)} to implement what can be seen as controller changes. 
In COP layers are added to the code that override certain behavior when this layer is activated for a specific tenant. 
The overriding methods can still call the original method allowing them to modify the output or implement a filter. 
Multiple layers can be active at once, doing their work in a predefined order.

Another common variation technique is \textit{dependency injection (DI)} \cite{walraven2011middleware} where one of the classes that implement an interface are injected by a provider. 
Walraven et al. mention in their Related Work section that Aspect Oriented Programming (another name for COP) proves more flexible in that way. 
In DI, no two 'modules' providing the same functionality can interact or extend each other, while in COP those implementations can work together.

\subsubsection{future work in variability}
