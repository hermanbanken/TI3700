In multi-tenancy \textit{variability} is a key concept. The term was first introduced in the car industry, where customers could choose certain \textit{variants} of chassis, engine and color \cite[p. 153]{kabbedijk2011variability}. 
In research on software engineering the concept was defined as ``the ability of a software system or artefact to be efficiently extended, changed, customized or configured for use in a particular context'' \cite{svahnberg2005taxonomy}.
Two keywords from this definition are customization and configuration. In a multi-tenant context configuration is preferred over customization \cite{sun2008software} as customization defines the process of reengineering an application, maintaining multiple branches and deploying these branches separately, while configuration can be done at run-time and does not require multiple instances or branches.

\subsubsection{Why is variability needed}
There are several reasons why variability is needed. 
First of all based on country, segment or branch different currencies, legislation and tax rules may apply. This is especially important in financial applications. 
Secondly different customers can require different functionality properties, layout options and/or quality of service (such as privacy and performance).

\subsubsection{Levels of variability}
Variability can be accomplished on different levels. 
Dependent on the scale of the application different patterns become be relevant. Large applications often consist of \textit{components} offering a specific \textit{service} and variability can be accomplished by dynamically swapping these components for different tenants, or by activating or deactivating specific components \cite{mietzner2008defining}. 
In smaller applications and within components different patterns become relevant. For example one can customize an application by using dependency injection \cite{walraven2011middleware} or context oriented programming \cite{truyen2012context}.

\subsubsection{variability techniques}
\subsubsection{variability modeling}
\subsubsection{future work in variability}