Multi-tenancy is a vague term, since it has never had an exact and official definition. During the last years, also due to the uprising of cloud applications, multi-tenancy gets researched and used more and more. In this section we give some recent definitions and we shortly describe the various types of multi-tenancy. We also discuss the relationship with the cloud and mention the challenges of multi-tenancy.

\subsection{Definitions}

In multi-tenant research, there are three important concepts: tenants, multi-tenancy, multi-user. In this section we show the variety of definitions of these concepts.

Bezemer and Zaidman~\cite{bezemer2010multi} define a tenant as an organizational entity which rents a multi-tenant SaaS solution (usually grouping a number of users). Krebs et al.~\cite{krebs2012architecture} have a more loose definition of a tenant; they define it as a group of users sharing the same view.

Based on these definitions, both more or less agree on the definition of a multi-tenant application: hardware resource sharing by offering one shared application and database instance to multiple tenants.

The difference between multi-tenancy and multi-user is also made clear by Bezemer and Zaidman~\cite{bezemer2010multi}. They state that in a multi-user application all users use the same application with limited configuration options, whereas a multi-tenancy application has more configuration options. However, all users use the same application, in both multi-user and multi-tenancy applications.

\subsection{Different types of multi-tenancy}

Krebs et al.~\cite{krebs2012architecture} describe various layers of sharing, in order of increasing benefits:
\begin{enumerate}
\item Sharing a data center (very limited benefits)
\item Virtualization, thus sharing a server (large overhead)
\item Middleware sharing (difficult isolation and overhead)
\item Multi-Tenant Application
\end{enumerate}

According to Bezemer and Zaidman~\cite{bezemer2010multi}, there are also various ways of implementing multi-tenancy: a shared application with a separate database, a shared application with a shared database with a separate table and a shared application, with a shared table. They considere the latter as pure multi-tenancy. Krebs et al.~\cite{krebs2012architecture} divide this into affinity and persistency, with affinity describing which server(s) handle which tenant(s) and with persistency describing the way of usage of the database (shared or separated).

Chang Jie Guo et al.~\cite{guo2007framework} suggest the implementation of a multi-tenancy enablement layer. This should create a separation between usage and resources, creating the required isolations and allowing customizations. This can be compared to the blueprint of Bezemer and Zaidman~\cite{bezemer2010multi}, since both approaches should have little impact on the single-tenant code.

\subsection{Multi-tenancy vs. the cloud}

Dillon et al.~\cite{dillon2010cloud} comprehensively elaborate on various aspects of cloud services. They mention the following service models: 
\begin{enumerate}
\item Software as a Service (example: Google Mail/Docs)
\item Platform as a Service (example: Google AppEngine)
\item Infrastructure as a Service (example: Amazon EC2)
\item Data as a Service (example: Google BigTable)
\end{enumerate}

Multi-tenancy falls mostly within the Software as a Service (SaaS) domain, as mentioned by Tsai et al.~\cite{tsai2010towards}. Multi-tenancy gains the most benefits within this model.

Dillon et al.~\cite{dillon2010cloud} also describe the several essential characteristics of cloud services, including resource pooling ('pooling' computing resources together in an effort to serve multiple consumers) and rapid elasticity (the consumption of resources can rapidly increase and the usage is not predictable upfront). These are both challenges for multi-tenant applications, as also mentioned by Krebs et al.~\cite{krebs2012architecture} and Bezemer and Zaidman~\cite{bezemer2010multi}.

\subsection{Challenges of multi-tenancy}

The multi-tenancy model has created two new security issues~\cite{dillon2010cloud}. Sharing resources on the same physical machine pose a danger to the data of the tenants. This falls within the isolation challenge. Another issue is reputation fate-sharing, since one might be sharing resources with possible criminal users, creating the possibility to (for example) get blacklisted.

Another challenge is the charging model. The costs of developing multi-tenancy can be very substantial for SaaS providers, because they need to re-design or re-develop single-tenant software, introduce new features for customization and enhance the security.

Bezemer and Zaidman~\cite{bezemer2010multi} also mention performance as a challenge, since multiple tenants are using the same hardware resources. Other challenges are scalability (usage of resources can suddenly increase), zero-downtime (the user expects the system to be online when he needs it) and maintenance (added complexity due to multi-tenancy could make code harder to maintain).